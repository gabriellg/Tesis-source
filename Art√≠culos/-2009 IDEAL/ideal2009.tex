\documentclass[runningheads]{llncs}

\usepackage{amssymb}
\usepackage{amsmath}
\setcounter{tocdepth}{3}
\usepackage{graphicx}
%\usepackage[spanish]{babel}
\usepackage{url}


\begin{document}

\mainmatter


\title{Formal model to integrate Multiagent Systems and Interactive Graphic Systems}
\titlerunning{Formal model to integrate Multiagent Systems and Interactive Graphic Systems}
%\toctitle{Formal model to integrate Multiagent Systems and Interactive Graphic Systems}


\author{Gabriel L\'opez Garc\'ia, Rafael Molina-Carmona and Javier Gallego-S\'anchez}
\authorrunning{Gabriel L\'opez \and Rafael Molina-Carmona \and Javier Gallego-S\'anchez}
%\tocauthor{G. L\'opez \and R. Molina \and J. Gallego}

\institute{Grupo de Inform\'atica Industrial e Inteligencia Artificial\\
Universidad de Alicante, Ap.99, E-03080, Alicante, Spain\\
\{glopez, rmolina, ajgallego\}@dccia.ua.es}



\maketitle


\begin{abstract}

A formal grammar-based model to integrate the essential characteristics of a Multiagent System
using the visualization given by an Interactive Graphic Systems is presented. This model adds
several advantages, such as the separation between the implementation of the system activity and
the hardware devices, or the easy reusability of components. To
illustrate the model, a practical case is presented. It implements a simulation of the fires in
forests caused by lightning.

\end{abstract}



%___________________________________________________________________________
\section{Introduction
\label{sec:introduction}}
%___________________________________________________________________________

The growing influence of Multi-Agent Systems (MAS) in several fields of research (sociology,
economics, artificial intelligence, ecology, and so on) has led to a significant evolution in its
development. On the other hand, the spectacular progress of the Interactive Graphic Systems (IGS)
has contributed to present the information in a more friendly way using new forms of analysis.
Videogames and the entertainment industry in general, have had a decisive influence on this
spectacular progress, that has arrived to the field of research \cite{Rhyne2000}. However, with few
exceptions, solutions to integrate MASs and IGSs are not proposed.

Agents are usually considered to have some generic characteristics that make them difficult to
model \cite{Gilbert2008}. Although there are some common strategies for implementing them, there is
a lack of a unified design pattern. As a consequence, some problems arise, such as the difficult
reproduction of the results provided by the MASs \cite{Axelrod1997}, the lack of a suitable visual
representation or the limited capabilities of interaction with the user and with the environment.
MASs could take advantage of the wide research carried out in this fields to develop IGSs.

There are many generic and specific work environments to develop MASs, but they seldom include
advanced visual features. For instance, in Sociology
\cite{Axelrod1997,Gilbert2008}, very specific solutions are used, generally
oriented to sociological studies, such as the movement of crowds
\cite{Ulicny2001,Reynolds2000}. In some cases, they use graphics to display statistics,
the density of agents within the environment or even the animation of the movements of agents
\cite{Reynolds2000}. Netlog \cite{Wilensky1999} is a useful tool to study highly complex systems.
It has developed some graphical features, but the agents are represented by Logo style triangles
just to allow the user to observe the movement agents and the population density.

Some applications of the MASs to Virtual Reality systems can also be found. One of these systems
proposes an implementation of a virtual world where objects react to several gestures of the user
\cite{Maes1995}. The work of \cite{Wachsmuth1995} uses the concepts of perception of the MASs
applied to virtual reality, where the agents react interactively with the user.

In the area of the entertainment industry, some MASs applied to games can also be found. In this
case, the agents are usually called bots \cite{Khoo2002} and they are programmed as characters of
the game, with objectives, strategies and actions. Programming is done with a specific script
language for each game, so that characters cannot easily be transferred between games. These
systems are so successful that they are being used in research for some specific cases
\cite{Rhyne2000}.

An increasing variety of generic development environments with a similar philosophy are also
emerging: they implement the most important features of agents, i.e., perception of the world,
communication, reaction, and so on \cite{Gilbert2008}. Some examples are Repast \cite{North2005}
and MASON \cite{Luke2005}.

As a conclusion, there are a variety of environments that implement the essential characteristics
of the MASs and, in some cases, they have some basic display functions. However, a model that
unifies the definition of MASs and the whole potential of IGS has not been proposed.

This paper describes an integral model based on grammars to develop complex environments that take
advantage of the MASs and the IGSs features. This system uses a descriptive language and discrete
events to specify all the features necessary to define an agent. It allows an easy interaction with
the user so that the initial conditions could be changed, the experiments can be reproduced and the
components of the scene are displayed in real time. It is also independent from the display and
interaction devices and it incorporates a physics engine. Finally, the agents can be easily reused.

The following section provides a description of the proposed model. In section 3, a case of study
is proposed with the aim of showing the use of the whole system. Finally, some conclusions and
possible lines of future work are presented.



%___________________________________________________________________________
\section{Model for Virtual Worlds Generation
\label{sec:model}}
%___________________________________________________________________________

In the proposed model, a scene is a set of dynamic and static elements. They are all represented by
a sequence of primitives an transformations defined in a geometric system $G$. A primitive is not
just understood as a draw primitive (e.g. a sphere) but an action on the geometric system that can
be a visual action or not. A transformation is a modification on the primitive behavior. They will
affect the set of primitives inside their scope of application.

The agents are the dynamic elements and they are made of activities and an optional set of internal
states. Each activity is a process that is executed as a reaction to a given event. The agents can
have a geometrical representation, defined using primitives and transformations, depending on their
internal state.

An event executes an activity. Its generation is independent from the physical device. They provide
the different ways of communication in a MAS.

Those are the main elements of the proposed system. Next, a formal mathematical model is presented,
to precisely define its features.


\begin{table}[h]
\begin{small}
\begin{center}
\begin{tabular}{|l|}

    \hline

    Rule 1.- \textbf{WORLD} $\rightarrow$ OBJECTS \\

    Rule 2.- \textbf{OBJECTS} $\rightarrow$ OBJECT $|$ OBJECT · OBJECTS \\

    Rule 3.- \textbf{OBJECT} $\rightarrow$ FIGURE $|$ TRANSFORMATION $|$ AGENT \\

    Rule 4.- \textbf{AGENT} $\rightarrow$ $a^d_{st}$ (OBJECTS), $a^d_{st} \in A^D_{ST}, d \in D, st \in ST$ \\

    Rule 5.- \textbf{TRANSFORMATION} $\rightarrow t$ (OBJECTS), $t \in T$ \\

    Rule 6.- \textbf{FIGURE} $\rightarrow$ $p^+$, $p \in P$ \\

    \hline

\end{tabular}
\end{center}
\end{small}
\label{tab:rules}
\caption{Rules of the grammar}
\end{table}


A string $w \in \Sigma^*$ is generated by the grammar $M$ (Table 1), if it can be
obtained starting with the initial symbol WORLD and using the given rules. The
language $L(M)$ is the set of all the strings
which can be generated by this method, so:
$L(M) = \lbrace w \in \Sigma^* \ | \ \text{WORLD} \stackrel{*}{\rightarrow} w \rbrace$.
This grammar is a context-independent grammar (or a type-2 grammar, according to the Chomsky
hierarchy). Therefore, there is a procedure which verifies if a scene is correctly described.

Apart from the language syntax, it is necessary to define the functionality of the strings, that is,
the semantics of the language. In our case, it is denoted using a denotational method, which defines
the meaning of the string through mathematical logic terms.



%___________________________________________________________________________
\subsubsection{Semantic Function for Primitives (Rule 6)
\label{sec:rule6}}

Rule 6 defines the syntax of a figure as a sequence of primitives. Primitive's semantics is defined
as a function $\alpha: P \rightarrow G$. Each symbol in the set $P$ carries out a primitive on a given geometric
system $G$. So, depending on the definition of the function $\alpha$ and on the geometric system $G$, the result
may be different. $G$ represents the actions which are run on a specific geometric system. An example
of geometric system are graphical libraries such as OpenGL or Direct3D. But, the function $\alpha$ has no
restrictions on the geometric system that can be applied.



%___________________________________________________________________________
\subsubsection{Semantic Function for Transformations (Rule 5)
\label{sec:rule5}}

The scope of a transformation is
limited by the symbols ``()''. Two functions are used to describe the semantics
of a transformation: $\beta: T \rightarrow G$ (it is carried out when the symbol ``('' is
processed), and $\delta: T \rightarrow G$ (it is carried out when the symbol ``)'' is found).
These two functions have the same features that the function $\alpha$, but they
are applied to the set of transformations $T$, using the same geometric system
$G$.

Given a string $w \in L(M)$, a new function $\varphi$ is defined to run a sequence of primitives $P$
and transformations $T$ in a geometric system $G$:


\begin{equation}
\begin{small}
    \varphi (w) = \left\{
    \begin{array}{ll}
        \alpha(w) & \mathit{if} \ \ w \in P  \\

        \beta(t); \varphi(v); \delta(t) & \mathit{if} \ \ w = t(v) \wedge v \in L(M) \wedge t \in T \\

        \varphi(s); \varphi(v)  & \mathit{if} \ \ w = s \cdotp v \wedge s, v \in L(M)
    \end{array}\right\}
\end{small}
\end{equation}



One of the most important features of this system is the independence from a specific graphics
system. The definition of the functions $\alpha$, $\beta$ and $\delta$ provides the differences in
behaviour, encapsulating the implementation details. Therefore, the strings developed to define
virtual worlds may be reused in other systems.



%___________________________________________________________________________
\subsubsection{Semantic Function for Agents (Rule 4)
\label{sec:rule4}}

The semantics of agents is a function which defines its evolution in time. It is called
\textit{evolution function} $\lambda$ and is defined as:
$\lambda: L(M) \times E^D \rightarrow L(M)$, where $E^D$ is the set of events
for the device $D$ (considering these devices as
any software or hardware process that sends events). By applying the function
$\lambda(w, e^{f})$, $w \in L(M)$ is transformed into another string $u$,
which allows the system to evolve. It has a different expression depending on
its evolution, but the general expression is defined as:

\begin{small}
\begin{equation}
    \lambda (a_{st}^{d}(v),e^{f})=
    \left\{
    \begin{array}{ll}
        u \in L(M) & \mathit{if}  \ \ f = d \\
        a_{st}^{d}(v)  & \mathit{if}  \ \ f \neq d
    \end{array}\right\}
\end{equation}
\end{small}

The result $u$ of the function may contain or not the own agent, it can generate other
event for the next frame or change the state `$st$' of the agent.

The function $\lambda$ can define a recursive algorithm, called
\textit{function of the system evolution} $\eta$. Given a set of
events $e^i, e^j, e^k, \dots, e^n$ (denoted as $e^v$, where $v \in D^+$)
and a string $w$, it describes the evolution of the system at a given point
in time. This algorithm also uses the operator $\prod_{\forall f \in v}$ which concatenates
strings.

\begin{small}
\begin{equation}
    \eta (w, e^v) = \left\{
    \begin{array}{ll}
        w   & \mathit{if}  \ \ w \in P  \\

        t(\eta (v, e^v))    & \mathit{if}  \ \  w = t(v)  \\

        \underset{\forall f \in v}{ \prod }(\lambda (a_{st}^d (\eta
            (y, e^v)), e^f))    & \mathit{if}  \ \ w = a_{st}^d(y) \\

        \eta (s, e^v) \cdot \eta (t, e^v)   & \mathit{if}  \
\  w = s \cdot t
    \end{array}\right\}
\end{equation}
\end{small}

For the visualization of an agent, it must be first converted into strings made up only
of primitives and transformations. This conversion is carried out by a type of function $\lambda$
called \textit{visualization function} $\theta: L(M) \times E^V \rightarrow L(E)$, where
$V \subseteq D$ are events used to create different views of the system, $E^V$ are events
created in the visualization process, and $L(E)$
is the language $L(M)$ without agents. This function is defined as:

\begin{small}
\begin{equation}
    \theta (a_{st}^d(v), e^f) =
    \left\{
    \begin{array}{ll}
        w \in L(E) & \ \ \mathit{if}  \ \ f = d \wedge d \in V \\
        \epsilon  & \ \ \mathit{if}  \ \ f \neq d
    \end{array}\right\}
\end{equation}
\end{small}

As with the function $\lambda$, an algorithm is defined for $\theta$. It returns a string $z \in
L(E)$, given a string $w \in L(M)$ and a set of events $e^v$, where $v \in V^+$ and $V \subseteq
D$. This function is called \textit{function of system visualization} $\pi$ and it is defined as:
$\pi: L(M) \times E^V \rightarrow L(E)$

\begin{small}
\begin{equation}
    \pi (w, e^v) = \left\{
    \begin{array}{ll}
        w   & \mathit{if}  \ \ w \in P^+  \\

        t(\pi (y, e^v))     & \mathit{if}  \ \  w = t(y)  \\

        \underset{\forall f \in v}{ \prod }(\theta (a_{st}^v (\pi
            (y, e^v)), e^f))    & \mathit{if}  \ \ w = a_{st}^v(y) \\

        \pi (s, e^v) \cdot \pi (t, e^v)    & \mathit{if}  \
            \  w = s \cdot t
    \end{array}\right\}
\end{equation}
\end{small}


%___________________________________________________________________________
\subsubsection{Semantic Functions for OBJECT, OBJECTS and WORLD (Rules 1,
2 and 3)
\label{sec:rules123}}

The semantic function of WORLD is a recursive function
which breaks down the string of the WORLD and converts it into substrings of OBJECTS. Then, these
substrings are in turn broken down into substrings of OBJECT. And for each substring of OBJECT,
depending on the type of the object, the semantic function of agent, transformation or primitive is
run. Primitives generated by agents without the visualization function represent the static
part of the system.

%___________________________________________________________________________
\subsection{Activity and Events
\label{sec:activity_events}}

In MAS some mechanisms must be established to model its activities.
These activities are run by agents and are activated by events.
Not all activities are run when an event is received, they can also be run when certain
conditions are satisfied.
The following event definition is established: {\itshape
$e_c^d$ is defined as an event of type $d \in D$ with data $e$, which is carried out only when the
condition $c$ is fulfilled. When there is no condition, the event is
represented by $e^d$.} Events may include information identifying who sent the message. So, it provides a generic
communication system that can implement FIPA or KMQL \cite{Genesereth1995}.


%___________________________________________________________________________
\subsection{Input Devices and Event Generators
\label{sec:input_devices}}

It is necessary to establish the independence between the system and the input devices
that generate events (hardware or software). So, the events needed to make
the system respond to a set of input devices must be defined.
A new function called \textit{event generator} is defined as: {\itshape
Let $C^d(t)$ be a function which creates events of type $d$ at the time instant $t$,
where $d \in D$ and $D$ is the set of event types which can be generated by the system. }

It is important to note that event generators encapsulate the device-dependent code.
They also can model the communication processes that exist in a MAS (agent-agent and
agent-environment communication).

The process which obtains the events produced by input devices and their associated
generators is defined as follows: {\itshape
Let $C^*$ be the set of all the event generators which are associated with input
devices and $E(C^*, t)$ the function that collects all the events from all the generators, then:}

\begin{small}
\begin{center}
\begin{tabular}{lll}
    $E(C^*, t) = \left\{
    \begin{array}{ll}
        e(z, C_i(t))   &  \mathit{if}  \ z = E(C^* - C_i, t) \\
        \epsilon   &  \mathit{if} \ C^* = \emptyset
    \end{array}
    \right\}$
    &, where \ &
    $e(z, e^i) = \left\{
    \begin{array}{ll}
        z \cdot e^i   &   \mathit{if} \ e^i \notin z \\
        z    &   \mathit{if} \ e^i \in z
    \end{array}
    \right\}$
\end{tabular}
\end{center}
\end{small}



%___________________________________________________________________________
\subsection{System Algorithm
\label{sec:system_algorithm}}

Once all the elements involved in the model to manage a MAS have been defined, the algorithm which carries out the
entire system can be established:

\begin{center}
\begin{tabular}{lll}
\begin{small}
\begin{tabular}{|ll|}
    \hline
    1. & $w = w_o$; $t = 0$ \\
    2. & $e^* = E(G^*, t)$ \\
    3. & $e^v =$ events of $e^*$ where $v \in V^+$ \\
    4. & $e^u = e^* - e^v$ \\
    5. & $w_{next} = \eta(w, e^u)$ \\
    6. & $v =  \pi(w, e^v)$ \\
    7. & $g = \varphi(v)$ \\
    8. & $w = w_{next}; \ \ t = t + 1$ \\
    9. & If $w = \epsilon$ then go to 11 \\
    10. & Go to 2 \\
    11. & End \\
\hline
\end{tabular}
\end{small}
     & &
\begin{small}
\begin{tabular}{p{0.5\linewidth}}
    $w_o$ is the intial string of the system.\\

    $e^*$ are all the events generated by the system in a frame $t$.\\

    $G^*$ = \{All the event generators which generate events of type $D$\}\\

    $D$ = \{All the possible events in the system\}\\

    $V$ = \{All the visual events\} where $V \subseteq D$\\

    $e^v$ are all the events from visual devices.\\

    $e^u$ are all the events from non-visual devices.\\

    $g$ is the output device.\\
\end{tabular}
\end{small}
\end{tabular}
\end{center}



Steps 2, 3 and 4 manage the system events. In step 5, the evolution algorithm is called to obtain the
string for the next frame. In steps 6 and 7, the visualization of the system is performed.
In step 8, the next iteration
is prepared. Step 9 checks if the current string satisfies the condition of completion: if the
following string is empty the algorithm ends (Step 11), otherwise the algorithm continues.

Step 5 and 6-7 can be parallelized because they do not share any data, it leads
to a faster system performance.



%___________________________________________________________________________
\section{Case of Study
\label{sec:case_study}}
%___________________________________________________________________________

This example is an application that simulates fires in forests caused by lightning \cite{John2007}.
The system basically consists of an agent who defines the forest. This agent
can create other agents that implement trees (with a given probability $g$)
and lightning (with a probability $f$). If a lightning is created in the same
position as a tree, it will burn as well as the trees around it.
To model this example, four elements are defined:
events, event generators, agents and graphic primitives.

Events are used to produce the necessary activity of system. Their aim is to run certain
activities of the agents that compose the scene. The events defined for this example are:

\begin{itemize}
    \item $t$: Event generated to increase the time since the previous event.

    \item $c$: Creates a tree at the position $(i, j)$ of the forest.

    \item $f$: Creates a bolt of lightning at position $(i, j)$.

    \item $e$: Eliminates the tree of the position $(i, j)$.

    \item $b$: Burns the tree at position $(i, j)$.

    \item $v$: Draws using a graphics library (e.g. OpenGL).
\end{itemize}


The next step is to define the event generators:

\begin{itemize}
    \item \textit{Time event generator} ($C_{time}$): It makes the animation of the system. Every time
    instant $t$, it generates an event $e^t$ to change the appearance of an agent.

    \item \textit{Forest event generator} ($C_{forest}$): It produces events to create trees ($e^{c}$)
    and lightning ($e^{f}$).

    \item \textit{Visualization event generator} ($C_{visualization}$): It captures the drawing orders and produces an
    event to send the elements to draw on the graphics system.
\end{itemize}


Table \ref{table2} shows the primitives and transformations that make up the scene.
The functions ${\alpha}$, ${\beta}$ and ${\delta}$ define them.

Table \ref{table4} shows the evolution function $\lambda$ and the graphical representation $\pi$
of the agents defined for this example. The agent defined for trees ($TR$) has three internal
states $st = \{s1, s2, s3 \}$: $s1$ is the state of growth, $s2$ corresponds to a grown tree and $s3$ to a burning tree.
This is an example of how an agent can have different representation depending on its internal state.
The function $Nbo$ sends burn events $e^b$ to all the neighbours trees of an agent, creating
a chain reaction (agent-agent communication). An example of agent-environment communication is made
between the forest and the generator $C_{forest}$.

\vspace{-0.4cm}

\begin{table}[h]
\begin{center}
\begin{small}
\begin{tabular}{lll}

    \begin{tabular}{|l|l|}
        \hline \itshape Primitive & \itshape Description\\
        \hline $TR$   & Draw a tree\\
        \hline $TR_b$ & Draw a burning tree\\
        \hline $FA$   & Draw a bolt of lightning\\
        \hline $BO$   & Draw a grid of NxN \\
        \hline
    \end{tabular}
     & &
    \begin{tabular}{|l|l|}
        \hline \itshape Transformations & \itshape Description\\
        \hline $D_{(i,j)}$ & Translate (i,j)\\
        \hline $S_{(s)}$ & Scale (n) units\\
        \hline
    \end{tabular}

\end{tabular}
\end{small}
\caption{\label{table2} (left) Definition of primitives, (right) Definition of transformations}
\end{center}
\end{table}


\vspace{-1.5cm}
\begin{table}[h]
\begin{center}
\begin{small}
\begin{tabular}{|p{0.08\linewidth}|p{0.15\linewidth}|p{0.73\linewidth}|}

    \hline
    \itshape Agent &
    \itshape Description &
    \itshape Function ${\lambda}$ and ${\pi}$\\
    \hline

    \ \ $BO$ &
    \ Forest &
    \begin{tabular}{lll}
        $\lambda(BO^{cfe}, e^{i})$ & $=$ & $\left\{
        \begin{array}{ll}
            TR_{s1}^{t=1} \cdot BO^{cfv}    & \ \ \ i = c \\
            FA^{t=1}   \cdot BO^{cfv}   & \ \ \ i = f \\
            BO^{cfe}            & \ \ \ i = e \\
            BO^{cfe}            & \ \ \ i \neq c,f,e
        \end{array}
        \right\}$ \\

        $\pi (BO^{v}, e^{i})$ & $=$ & $\left\{
        \begin{array}{ll}
            BO          & \ \ \   i = v \\
            \epsilon    & \ \ \   i \neq v
        \end{array}
        \right\}$
    \end{tabular}   \\

    \hline

    \ \ $TR$ &
    \ Tree &
    \begin{tabular}{lll}
        $\lambda(TR_{st}^x, e^{i})$ & $=$ & $\left\{
        \begin{array}{ll}
            TR_{s1}^{t+1}       & i = t \wedge t+1 \leqslant N \wedge s = s1 \\
            TR_{s2}^{b}         & i = t \wedge t+1 > N \wedge s = s1 \\
            TR_{s3}^{t=1} \wedge \Delta Nbo^{b}         & i = b > N \wedge s = s2 \\
            TR_{s3}^{t+1}       & i = t \wedge s = s3 \\
            \Delta e^{e}        & i = t \wedge  t+1 > N  \wedge s = s3 \\
            TR_{st}^{t}         & i \neq t \wedge   i \neq b \\
        \end{array}
        \right\}$ \\

        $\pi (TR_{st}^{v}, e^{i})$ & $=$ & $\left\{
        \begin{array}{ll}
            D_{(i, j)}( S_{(n)} (TR) )    & \ \ \   i = v \wedge st = s1 \\
            D_{(i, j)}( TR )        & \ \ \   i = v \wedge st = s2 \\
            D_{(i, j)}( S_{(-n)} (TR) )   & \ \ \   i = v \wedge st = s3 \\
            \epsilon            & \ \ \   i \neq v
        \end{array}\right\}$
    \end{tabular}   \\


    \hline

    \ \ $FA$ &
    \ Lightning &
    \begin{tabular}{lll}
        $\lambda (FA^{t}, e^{i})$ & $=$ & $\left\{
        \begin{array}{ll}
            FA^{t+1}        & \ \ \  i = t \wedge t+1 \leqslant N \\
            \Delta e^{b}        & \ \ \  i = t \wedge t+1 > N \\
            FA^{t}          & \ \ \  i \neq t
        \end{array}
        \right\}$ \\

        $\pi (FA^{v}, e^{i})$ & $=$ & $\left\{
        \begin{array}{ll}
            D_{(i, j)}(FA)          & \ \ \   i = v \\
            \epsilon            & \ \ \   i \neq v
        \end{array}
        \right\}$
    \end{tabular}   \\

    \hline
\end{tabular}
\end{small}
\caption{\label{table4} Agents defined for this example}
\end{center}
\end{table}






\vspace{-1cm}
%___________________________________________________________________________
\section{Conclusions and Future Work
\label{sec:conclusions}}
%___________________________________________________________________________

In this paper a proposal to unify the most relevant features of MASs and IGSs has been presented.
The proposed model uses a context-free language to define the elements of the system. Although
further work needs to be done, the use of a descriptive language like this seems to have several
advantages.

Firstly, the definition of the scene is reusable and independent from the platform thanks to the
use of the semantic functions. The only task to reuse the scene in a new graphical system is to
adapt the semantic functions. This functions can be adapted to the graphical requirements of the
application, so the display can be as complex as desired, from schematic representations to
realistic scenes for virtual reality.

The use of event generators also makes the interaction with the user independent from the hardware.
This feature allows an easy introduction of new interaction devices, just changing the device that
activates the event generator, but keeping the event itself invariable.

The event generators are also used to implement an important feature of a MAS: the communication,
both agent-agent and agent-environment. The communication between agents is achieved through events
generated by the agents. The incorporation of a event generator acting as a physics engine (it
would generate events to inform about collisions, gravity and so on) allows the communication
between the environment and the agents.

In this model, the semantic functions and the event generators describe the general behaviour of
the system, acting as a sort of compiler. An instance of the system, that is, a particular scene,
is defined by a string that is compiled by the system. The use of a string allows the reproduction
of the experiments just recompiling the scene.

Some of this features are illustrated through the simple example that has been presented.
Nevertheless, this case is very simple and has some limitations, so it will be extended and
improved in the future. Furthermore, the model will be applied to other problems to validate its
features. It also opens new prospects in the future such as the development of agents that are made
up of other agents or the execution of actions associated to events with a given probability. In
this field, new possibilities such as probabilistic learning strategies or genetic algorithms will
be considered.





\begin{thebibliography}{0}

\bibitem{Axelrod1997}
\textsc{Robert Axelrod}
\newblock Advancing the Art of Simulation in the Social Sciences
\newblock \emph{Simulating Social Phenomena}, Springer, 1997

\bibitem{Genesereth1995}
\textsc{Michael R. Genesereth, Steven P. Ketchpel}
\newblock Software Agents
\newblock \emph{Communications of the ACM} 1995

\bibitem{Gilbert2008}
\textsc{Nigel Gilbert}
\newblock Agent-Based Models
\newblock \emph{SAGE Publications} 2008

\bibitem{John2007}
\textsc{John H.Miller}
\newblock Complex Adaptative Systems
\newblock \emph{Princeton University Press} 2007

\bibitem{Khoo2002}
\textsc{Aaron Khoo, Robert Zubek}
\newblock Applying Inexpensive AI Techniques to Computer Games
\newblock \emph{IEEE Intelligent Systems} July-August 2002

\bibitem{Luke2005}
\textsc{Sean Luke, Claudio Cioffi-Revilla, Liviu Panait, and Keith Sullivan}
\newblock MASON: A New Multi-Agent Simulation Toolkit
\newblock \emph{Vol. 81, SAGE Journals} 2005

\bibitem{Maes1995}
\textsc{Pattie Maes, Trevor Darrell, Bruce Blumberg, Alex Pentland}
\newblock The ALIVE System:Wireless, Full-body Interaction with Autonomous Agents,
\newblock \emph{ACM Multimedia Systems, Vol.5, No.2, pp.105-112} 1997

\bibitem{North2005}
\textsc{M.J. North, T.R. Howe, N.T. Collier, J.R. Vos}
\newblock The Repast Simphony Runtime System
\newblock \emph{Conference on Generative Social Processes, Models, and Mechanisms, Proceedings of the Agent} 2005

\bibitem{Reynolds2000}
\textsc{Craig Reynolds}
\newblock Interaction with Groups of Autonomous Characters
\newblock \emph{Game Developers Conference} 2000

\bibitem{Rhyne2000}
\textsc{Theresa-Marie Rhyne}
\newblock Computer Games’Influence on Scientific and Information Visualization
\newblock \emph{Entertainment Computing} 2000

\bibitem{Ulicny2001}
\textsc{B. Ulicny, D. Thalmann}
\newblock Crowd simulation for interactive virtual environments and VR training systems
\newblock \emph{Computer Animation and Simulation, Springer} 2001

\bibitem{Wilensky1999}
\textsc{Wilensky, U., NetLogo}
\newblock User Manual, 1999

\bibitem{Wachsmuth1995}
\textsc{Ipke Wachsmuth, Yong Cao}
\newblock Interactive Graphics Design with Situated Agents
\newblock \emph{Graphics and Robotics} 1995

\bibitem{Davis1994}
\textsc{D.Martin, R.Sigal, E.J.Weyuker}:
\emph{Computability, Complexity, and Languages, Fundamentals of Theoretical Computer Science},
2nd~ed, Elsevier Science 1994

\end{thebibliography}



\end{document}
